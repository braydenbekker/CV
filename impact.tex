%%%%%%%%%%%%%%%%%%%%%%%%%%%%%%%%%%%%%%%%%
% a0poster Portrait Poster
% LaTeX Template
% Version 1.0 (22/06/13)
%
% The a0poster class was created by:
% Gerlinde Kettl and Matthias Weiser (tex@kettl.de)
% 
% This template has been downloaded from:
% http://www.LaTeXTemplates.com
%
% License:
% CC BY-NC-SA 3.0 (http://creativecommons.org/licenses/by-nc-sa/3.0/)
%
%%%%%%%%%%%%%%%%%%%%%%%%%%%%%%%%%%%%%%%%%

%----------------------------------------------------------------------------------------
%	PACKAGES AND OTHER DOCUMENT CONFIGURATIONS
%----------------------------------------------------------------------------------------

\documentclass[a0,portrait]{a0poster}

\usepackage{multicol} % This is so we can have multiple columns of text side-by-side
\columnsep=100pt % This is the amount of white space between the columns in the poster
\columnseprule=3pt % This is the thickness of the black line between the columns in the poster

\usepackage[svgnames]{xcolor} % Specify colors by their 'svgnames', for a full list of all colors available see here: http://www.latextemplates.com/svgnames-colors

\usepackage{times} % Use the times font
%\usepackage{palatino} % Uncomment to use the Palatino font

\usepackage{graphicx} % Required for including images
\graphicspath{{figures/}} % Location of the graphics files
\usepackage{booktabs} % Top and bottom rules for table
\usepackage[font=small,labelfont=bf]{caption} % Required for specifying captions to tables and figures
\usepackage{amsfonts, amsmath, amsthm, amssymb} % For math fonts, symbols and environments
\usepackage{wrapfig} % Allows wrapping text around tables and figures

\begin{document}

%----------------------------------------------------------------------------------------
%	POSTER HEADER 
%----------------------------------------------------------------------------------------

% The header is divided into two boxes:
% The first is 75% wide and houses the title, subtitle, names, university/organization and contact information
% The second is 25% wide and houses a logo for your university/organization or a photo of you
% The widths of these boxes can be easily edited to accommodate your content as you see fit

\begin{minipage}[b]{0.75\linewidth}
\veryHuge \color{NavyBlue} \textbf{My Research Impact} \color{Black}\\ % Title
\Huge\textit{An Overview of the he Progress and Impact of My Research}\\[2cm] % Subtitle
\huge \textbf{Brayden Bekker}\\[0.5cm] % Author(s)
\huge Brigham Young University Department of Physics and Astronomy\\[0.4cm] % University/organization
\Large \texttt{brayden.bekker@byu.edu} --- 1 (801) 931 7583\\
\end{minipage}
%
\begin{minipage}[b]{0.25\linewidth}
\includegraphics[width=16cm]{BYU.png}\\
\end{minipage}

\vspace{1cm} % A bit of extra whitespace between the header and poster content

%----------------------------------------------------------------------------------------

\begin{multicols}{2} % This is how many columns your poster will be broken into, a portrait poster is generally split into 2 columns

%----------------------------------------------------------------------------------------
%	ABOUT ME
%----------------------------------------------------------------------------------------

\color{DarkSlateGray} % DarkSlateGray color for the rest of the content

\section*{About Me}
I am an undergraduate at Brigham Young University studying Physics. I joined the Materials Simulation Group (MSG) My freshman year after Dr. Hart presented the group's research in Physcs 191. During my time in the group I have: developed programming skills in C++, and python, learned good coding practices, and learned how to work together with a team to find solutions to important problems.  

%----------------------------------------------------------------------------------------
%	ABSTRACT
%----------------------------------------------------------------------------------------
\color{Navy} % Navy color for the abstract

\begin{abstract}
One of the central problems in materials research is accurately predicting the properties of an alloy. Computational materials discovery has the potential to open the floodgates for human progression. Our society, built upon the materials discoveries of the past, lies in wait for the discovery of the next material that will change the world. 
\end{abstract}
%----------------------------------------------------------------------------------------
%	INTRODUCTION
%----------------------------------------------------------------------------------------

\color{SaddleBrown} % SaddleBrown color for the introduction

\section*{Introduction}
In the past metallurgists made there discoveries through an exhaustive trial and error process which eventually resulted in a happy mistake. 

\begin{center}\vspace{1cm}
\includegraphics[width=0.6\linewidth]{RoyPlunkett.png}
\end{center}\vspace{1cm}

Roy Plunkett was trying to create a new refrigerant when a buildup of a waxy material started to form on the inside of the container of CFC gasses he was working with. Teflon soon followed.
With computers came computational methods which remove much of the guess work, however, these modern processes of materials prediction are still too slow. 

\begin{center}\vspace{1cm}
\includegraphics[width=0.6\linewidth]{materialscooking.png}
\end{center}\vspace{1cm}

Imagine for me a stocked pantry, to create a dessert by trial and error would be a time consuming disaster. A cookbook provides necessary directions. If you know the recipe than the desired result can be achieved. If you imagine now, a materials pantry stocked with every possible configuration of elements the number of combinations are infinite. The existing computational methods like DFT (Density Functional Theory) act as materials cookbooks. While the existing "materials cookbooks" remove the guess work they have a time consuming method for searching the "table of contents" to continue or example. DFT makes accurate predictions but at high computational cost.    

\begin{center}\vspace{1cm}
\includegraphics[width=0.6\linewidth]{MBTR.png}
\end{center}\vspace{1cm}

My current research is with the recently proposed Many Body Tensor Representation (MBTR) for atomic systems. This machine learning approach uses the existing materials "cookbooks" to train and then make out-of-sample predictions. My project researches the ability of MBTR to replace the existing materials "cookbooks" using machine learning to predict new materials at a faster rate.

%----------------------------------------------------------------------------------------
%	OBJECTIVES
%----------------------------------------------------------------------------------------

\color{DarkSlateGray} % DarkSlateGray color for the rest of the content

\section*{Main Objectives}

\begin{enumerate}
\color{Green} % DarkSlateGray color for the rest of the content
\item Search the Interatomic Potentials Repository for binary/ternary systems with EAM potentials
\item Use Enumlib to get the derivative superstructures for each system
\item For each system from FCC, BCC, and HCP choose 2500 derivative structures: the first 500 and 2000 additional structures from 10-12 atom cells.  

\color{DarkSlateGray} % DarkSlateGray color for the rest of the content

\item Relax the atomic positions and compute energies per atom and enthalpy.
\item Plot the convex hull and verify with AFLOW that results are similar (EAM potentials will not match results, here we are only looking for general similarities.)
\item Save the data for each system to a HDF5 file for ease of data access.
\item Use results to train and predict in MBTR 
\end{enumerate}

%----------------------------------------------------------------------------------------
%	MATERIALS AND METHODS
%----------------------------------------------------------------------------------------

%\section*{Materials and Methods}

%------------------------------------------------

%\subsection*{Mathematical Section}

%Nulla vel nisl sed mauris auctor mollis non sed. 

%\begin{equation}
%E = mc^{2}
%\label{eqn:Einstein}
%\end{equation}

%Curabitur mi sem, pulvinar quis aliquam rutrum. (1) edf (2)
%, $\Omega=[-1,1]^3$, maecenas leo est, ornare at. $z=-1$ edf $z=1$ sed interdum felis dapibus sem. $x$ set $y$ ytruem. 
%Turpis $j$ amet accumsan enim $y$-lacina; 
%ref $k$-viverra nec porttitor $x$-lacina. 

%Vestibulum ac diam a odio tempus congue. Vivamus id enim nisi:

%\begin{eqnarray}
%\cos\bar{\phi}_k Q_{j,k+1,t} + Q_{j,k+1,x}+\frac{\sin^2\bar{\phi}_k}{T\cos\bar{\phi}_k} Q_{j,k+1} &=&\nonumber\\ 
%-\cos\phi_k Q_{j,k,t} + Q_{j,k,x}-\frac{\sin^2\phi_k}{T\cos\phi_k} Q_{j,k}\label{edgek}
%\end{eqnarray}
%and
%\begin{eqnarray}
%\cos\bar{\phi}_j Q_{j+1,k,t} + Q_{j+1,k,y}+\frac{\sin^2\bar{\phi}_j}{T\cos\bar{\phi}_j} Q_{j+1,k}&=&\nonumber \\
%-\cos\phi_j Q_{j,k,t} + Q_{j,k,y}-\frac{\sin^2\phi_j}{T\cos\phi_j} Q_{j,k}.\label{edgej}
%\end{eqnarray} 

%Nulla sed arcu arcu. Duis et ante gravida orci venenatis tincidunt. Fusce vitae lacinia metus. Pellentesque habitant morbi. $\mathbf{A}\underline{\xi}=\underline{\beta}$ Vim $\underline{\xi}$ enum nidi $3(P+2)^{2}$ lacina. Id feugain $\mathbf{A}$ nun quis; magno.

%----------------------------------------------------------------------------------------
%	RESULTS 
%----------------------------------------------------------------------------------------

%\section*{Results}

%Donec faucibus purus at tortor egestas eu fermentum dolor facilisis. Maecenas tempor dui eu neque fringilla rutrum. Mauris \emph{lobortis} nisl accumsan. Aenean vitae risus ante.
%
%\begin{wraptable}{l}{12cm} % Left or right alignment is specified in the first bracket, the width of the table is in the second
%\begin{tabular}{l l l}
%\toprule
%\textbf{Treatments} & \textbf{Response 1} & \textbf{Response 2}\\
%\midrule
%Treatment 1 & 0.0003262 & 0.562 \\
%Treatment 2 & 0.0015681 & 0.910 \\
%Treatment 3 & 0.0009271 & 0.296 \\
%\bottomrule
%\end{tabular}
%\captionof{table}{\color{Green} Table caption}
%\end{wraptable}
%\%
%Phasellus imperdiet, tortor vitae congue bibendum, felis enim sagittis lorem, et volutpat ante orci sagittis mi. Morbi rutrum laoreet semper. Morbi accumsan enim nec tortor consectetur non commodo nisi sollicitudin. Proin sollicitudin. Pellentesque eget orci eros. Fusce ultricies, tellus et pellentesque fringilla, ante massa luctus libero, quis tristique purus urna nec nibh.

%Nulla ut porttitor enim. Suspendisse venenatis dui eget eros gravida tempor. Mauris feugiat elit et augue placerat ultrices. Morbi accumsan enim nec tortor consectetur non commodo. Pellentesque condimentum dui. Etiam sagittis purus non tellus tempor volutpat. Donec et dui non massa tristique adipiscing. Quisque vestibulum eros eu. Phasellus imperdiet, tortor vitae congue bibendum, felis enim sagittis lorem, et volutpat ante orci sagittis mi. Morbi rutrum laoreet semper. Morbi accumsan enim nec tortor consectetur non commodo nisi sollicitudin.

%\begin{center}\vspace{1cm}
%\includegraphics[width=0.8\linewidth]{placeholder}
%\captionof{figure}{\color{Green} Figure caption}
%\end{center}\vspace{1cm}

%In hac habitasse platea dictumst. Etiam placerat, risus ac.

%Adipiscing lectus in magna blandit:

%\begin{center}\vspace{1cm}
%\begin{tabular}{l l l l}
%\toprule
%\textbf{Treatments} & \textbf{Response 1} & \textbf{Response 2} \\
%\midrule
%Treatment 1 & 0.0003262 & 0.562 \\
%Treatment 2 & 0.0015681 & 0.910 \\
%Treatment 3 & 0.0009271 & 0.296 \\
%\bottomrule
%\end{tabular}
%\captionof{table}{\color{Green} Table caption}
%\end{center}\vspace{1cm}

%Vivamus sed nibh ac metus tristique tristique a vitae ante. Sed lobortis mi ut arcu fringilla et adipiscing ligula rutrum. Aenean turpis velit, placerat eget tincidunt nec, ornare in nisl. In placerat.

%\begin{center}\vspace{1cm}
%\includegraphics[width=0.8\linewidth]{placeholder}
%\captionof{figure}{\color{Green} Figure caption}
%\end{center}\vspace{1cm}

%----------------------------------------------------------------------------------------
%	CONCLUSIONS
%----------------------------------------------------------------------------------------

%\color{SaddleBrown} % SaddleBrown color for the conclusions to make them stand out

%\section*{Conclusions}

%\begin{itemize}
%\item Pellentesque eget orci eros. Fusce ultricies, tellus et pellentesque fringilla, ante massa luctus libero, quis tristique purus urna nec nibh. Phasellus fermentum rutrum elementum. Nam quis justo lectus.
%\item Vestibulum sem ante, hendrerit a gravida ac, blandit quis magna.
%\item Donec sem metus, facilisis at condimentum eget, vehicula ut massa. Morbi consequat, diam sed convallis tincidunt, arcu nunc.
%\item Nunc at convallis urna. isus ante. Pellentesque condimentum dui. Etiam sagittis purus non tellus tempor volutpat. Donec et dui non massa tristique adipiscing.
%\end{itemize}

%\color{DarkSlateGray} % Set the color back to DarkSlateGray for the rest of the content

%----------------------------------------------------------------------------------------
%	FORTHCOMING RESEARCH
%----------------------------------------------------------------------------------------

%\section*{Forthcoming Research}

%Vivamus molestie, risus tempor vehicula mattis, libero arcu volutpat purus, sed blandit sem nibh eget turpis. Maecenas rutrum dui blandit lorem vulputate gravida. Praesent venenatis mi vel lorem tempor at varius diam sagittis. Nam eu leo id turpis interdum luctus a sed augue. Nam tellus.

 %----------------------------------------------------------------------------------------
%	REFERENCES
%----------------------------------------------------------------------------------------

%\nocite{*} % Print all references regardless of whether they were cited in the poster or not
%\bibliographystyle{plain} % Plain referencing style
%\bibliography{sample} % Use the example bibliography file sample.bib

%----------------------------------------------------------------------------------------
%	ACKNOWLEDGEMENTS
%----------------------------------------------------------------------------------------

%\section*{Acknowledgements}

%Etiam fermentum, arcu ut gravida fringilla, dolor arcu laoreet justo, ut imperdiet urna arcu a arcu. Donec nec ante a dui tempus consectetur. Cras nisi turpis, dapibus sit amet mattis sed, laoreet.

%----------------------------------------------------------------------------------------

\end{multicols}
\end{document}